\begin{figure}
	\begin{subfigure}{\textwidth}
		\begin{tikzpicture}[scale=1,samples=20,
							dot/.style={circle,inner sep=1pt,fill}]
			\coordinate (O) at (0,0);
			\coordinate (G1) at (0,0);
			\coordinate (G2) at (0,5);
			\coordinate (G3) at (10,5);
			\coordinate (G4) at (10,0);

			\tikzmath{
				% Global
				\xmax=5;	\ymax=3;
				% Graph 1
				\gxmax=\xmax-.5;	\gymax=\ymax-.5;
				%Graph 2
				\ggxmax=\xmax+2;	\ggymax=\ymax;	\tm=6.885;
			}
			\coordinate (Xmax) at (\xmax,0);
			\coordinate (Ymax) at (0,\ymax);

			% Graph 1: Membrane potential evolution
				\draw [-latex] ($(G1)+(-.1,0)$) -- ($(G1)+(Xmax)$) node [right] {t};
				\draw [-latex] ($(G1)+(0,-.1)$) -- ($(G1)+(Ymax)$) node [left] {$V_t$};
				% Curve of membrane potential
				\draw [color=gray!50,domain=0:\gxmax] plot (\x,{\gymax*(1-exp(-\x))}) node [color=black,right] {Unnoisy $V_t$};
				\draw [domain=0:\gxmax] plot (\x,{\gymax*(1-exp(-\x))+.5*rand}) node [right] {$V_t$};
				% Poisson process points
				\coordinate (P1) at (rnd,0);
				\coordinate (P2) at (rnd+1,0);
				\coordinate (P3) at (rnd+2,0);
				\coordinate (P4) at (rnd+3,0);
				\coordinate (P5) at (rnd+4,0);
				\draw (P1) node [circle,inner sep=1pt,fill] {};
				\draw (P2) node [circle,inner sep=1pt,fill] {};
				\draw (P3) node [circle,inner sep=1pt,fill] {};
				\draw (P4) node [circle,inner sep=1pt,fill] {};
				\draw (P5) node [circle,inner sep=1pt,fill] {};
			% Arrow
				\draw [-latex] (2.5,3.5) -- (6,4.5);
			% Graph 2: rejection sampling
				\begin{scope}[shift={(2.5,5)}]
					\coordinate (O) at (0,0);
					\draw [name path=x-absciss, -latex] (-0.1,0) -- (\ggxmax,0) node [right] {$x=V_t$};
					\draw [name path=y-absciss, -latex] (0,-0.1) -- (0,\ggymax) node [left] {f(x),$\text{\~f}^i(x)$};

					% \draw (0,\ggymax-.5) -- (7,\vmax) node [right] {$g(t)$};
					\draw [name path=curve,domain=0:\tm] plot (\x,{(-(\x-2)*cos((\x-2) r)-2*cos(2 r))/2}) node [above right] {$f(x)$};

					\draw (6,2) -- (7.5,2) node [right] {rejection zone};
					\draw (6,1) -- (7.5,1) node [right] {acceptation zone};

					\path [name path=xabsciss] (0,0) -- (\ggxmax,0);

					\path [name path=vert] (1,0) -- (1,\ggymax);
					\path [name intersections={of=vert and curve,by=E1}];
					\coordinate (E) at ($(E1)+(0,.2)$);
					\path [name path=vert] (2.5,0) -- (2.5,\ggymax);
					\path [name path=horz] ($(O)!(E)!(0,\ggymax)$) -- ($(\ggxmax,0)!(E)!(\ggxmax,\ggymax)$);
					\path [name intersections={of=vert and horz,by=E2}];
					\path [name intersections={of=vert and curve,by=E3}];
					\draw ($(O)!(E)!(0,\ggymax)$) -- (E2) node [midway,above] {$\text{\~f}^1(t)$};

					\path [name path=vert5] (5,0) -- (5,\ggymax);
					\path [name intersections={of=vert5 and curve,by=E5}];
					\path [name path=horz5] (E5) -- ($(O)!(E5)!(0,\ggymax)$);
					\path [name intersections={of=horz5 and vert,by=E6}];
					\draw (E6) -- (E5) node [midway,above] {$\text{\~f}^2(t)$};
					\draw [dotted] (E6) -- ($(O)!(E6)!(\ggxmax,0)$);

					\coordinate (E7) at ($(E5)+(0,.3)$);
					\path [name path=vert7] (\tm,0) -- (\tm,\ggymax);
					\path [name path=horz7] (E7) -- ($(\ggxmax,0)!(E7)!(\ggxmax,\ggymax)$);
					\path [name intersections={of=vert7 and horz7,by=E8}];
					\draw [dotted] (E7) -- ($(O)!(E7)!(\ggxmax,0)$);
					\draw (E7) -- (E8) node [midway,above] {$\text{\~f}^3(t)$};
					\draw [dotted] (E8) -- (\tm,0);
					
					\draw (.1,.8) node [dot] {};
					\draw (1.25,.1) node [dot] {};
					\draw (2.3,.5) node [dot] {};

					\draw (3.4,1.6) node [dot] {};
					\draw (3,.2) node [dot] {};
					\draw (4.8,1.7) node [dot] {};
					
					\draw (5.2,1.2) node [dot] {};
					\draw (6.1,2.1) node [dot] {};
				\end{scope}
				% Arrow
				\draw [-latex] (6,4.5) -- (12.5,3.5) node [midway,above] {accepted};
				\draw [-latex] (6,4.5) -- (6,-2.5) node [midway,above] {rejected};
				%Graph 3: effect of accepted spike
				\begin{scope}[shift={(10,0)}]
					\draw [-latex] (-.1,0) -- (\xmax,0) node [right] {t};
					\draw [-latex] (0,-.1) -- (0,\ymax) node [left] {$V_t$};
					% Curve of membrane potential
					\draw [color=gray!50,domain=0:\gxmax/2] plot (\x,{\gymax*(1-exp(-\x))});
					\draw [domain=0:\gxmax/2] plot (\x,{\gymax*(1-exp(-\x))+.5*rand});

					\draw [color=gray!50,domain=0:\gxmax/2] plot (\x+\gxmax/2,{\gymax*(1-exp(-\x))}) node [color=black,right] {Unnoisy $V_t$};
					\draw [domain=0:\gxmax/2] plot (\x+\gxmax/2,{\gymax*(1-exp(-\x))+.5*rand}) node [right] {$V_t$};
					% Poisson process points
					\coordinate (P1) at (rnd,0);
					\coordinate (P2) at (rnd+1,0);
					\coordinate (P3) at (rnd+2,0);
					\coordinate (P4) at (rnd+3,0);
					\coordinate (P5) at (rnd+4,0);
					\draw (P1) node [circle,inner sep=1pt,fill] {};
					\draw (P2) node [circle,inner sep=1pt,fill] {};
					\draw (P3) node [circle,inner sep=1pt,fill] {};
					\draw (P4) node [circle,inner sep=1pt,fill] {};
					\draw (P5) node [circle,inner sep=1pt,fill] {};
				\end{scope}
				% Graph 4: Group choice
				\begin{scope}[shift={(2.5,-7)}]
					\coordinate (O) at (0,0);
					\tikzmath{
								\tmax=7.5;	\fmax=5;
								\tm=6.885;	\vmax=\fmax-.5;
					}

					\draw [name path=x-absciss, -latex] ($(O)+(-0.1,0)$) -- ($(O)+(\tmax,0)$) node [right] {t};
					\draw [name path=y-absciss, -latex] ($(O)+(0,-0.1)$) -- ($(O)+(0,\fmax)$) node [left] {$f^i(t),g(t)$};

					\draw ($(O)+(0,\fmax-.5)$) -- ($(O)+(7,\vmax)$) node [above right] {$g(t)$};
					\draw [domain=0:\tm] plot (\x,{(-(\x-2)*cos((\x-2) r)-2*cos(2 r))/2}) node [above right] {$f^1(t)$};
					\draw [domain=-1:\tm-1] plot (\x+1,{(-(\x-2)*cos((\x-2) r)-2*cos(2 r))/2+2}) node [below right] {$f^2(t)$};

					\draw (6,4.3) -- (7.5,4.3) node [right] {rejection zone};
					\draw (6,1) -- (7.5,1) node [right] {acceptation zone of $f^1(t)$};
					\draw (6,2) -- (7.5,2) node [right] {acceptation zone of $f^2(t)$};
					%\draw (.1,1.5) node [dot] {} -- (-.5,1.5) node [left] {Randomy generated};
					\foreach \x in {0,...,8}{
						\tikzmath{
									\t=5.9*rnd;	\y=(\vmax-.2)*rnd;
						}
						\draw (\t+.1,\y+.1) node [dot] {};
					};
				\end{scope}
				\draw [-latex] (6,-2.5) -- (2.5,-.5) node [midway, above] {Choice of a new neuron};
		\end{tikzpicture}
		
		\label{eq:spde}
	\end{subfigure}\\
	\label{fig:rejectaccept}
\end{figure}