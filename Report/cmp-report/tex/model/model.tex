	In this study we are interrested in a large (possibly infinite) network of interacting integrate and fire neurons. \cite{lewis_dynamics_2003} and \cite{ostojic_synchronization_2009} proposed an equation describing the evolution in time of the potential $X_i$ of the $i^{th}$ neuron in a network of N
		\begin{equation}
			\hspace*{-2cm} \begin{cases}
				\frac{d}{dt}X_i(t)=-\lambda X_i(t)+\frac{\alpha}{N}\sum_j\sum_i\delta_0(t-\tau_k^j)+\frac{\beta}{N}\sum_{j\neq i}X_j(t)+I_i^{ext}(t)+\sigma\mu_i(t) &\quad\text{if }X_i<X_T\\
				X_i(t^+)=X_R & \quad\text{if }X_i\text{ reaches }X_T\text{ at time t}\\
			\end{cases}
		\end{equation}
	\cite{ostojic_synchronization_2009} and \cite{delarue:hal-00747565} gives more detail about the physical signification of the terms in the equation. As an overview, $X_i(t)$ is the function associated with the evolution in time of the potential of the membrane of the neuron i, $I_i^{ext}(t)+\sigma\mu_i(t)$ is the effect of electrical currents outside of the studied network (mean value + gaussian white noise), $\frac{\alpha}{N}\sum_j\sum_i\delta_0(t-\tau_k^j)$ 

	\section{Motivations}
	Here we are interested only on the spiking behaviour of the system, and so the mean field equation considered thereafter in this document is a simplified form of the one presented just above, as some terms are irreleveant to the question.
	In this paper we are going to focus on what sets of parameters favors the apparition of a blowing up of the system. The influence of the interaction term is a big part of the study, but the influence of other parameters, such as the b function, the noise, and the topology of the network are also looked at.\\
	In the deterministic case, the blow up is quite easily defined by the limit to a time value $t_b$ of the variations in the spike rate equals to infinity, in other words
	\begin{equation}
		\lim_{t\rightarrow t_b}\frac{de}{dt}=\infty
	\end{equation}
	This behaviour of the PDE has been described in \cite{delarue:hal-00747565}, while others have been interrested in this exact phenomenon from a PDE perspective. Indeed, systems ruled by this kind of equations do not automatically blow-up; actually some conditions on the parameters must be met in order to observe this phenomenon.

	\section{Mean field equation}
	That being said, we can now pose the real equation under study in this report.\\
	\[
		X_t^i=X_0^i+\int_0^tb(X_s^i)ds+\sigma W_t^i+\sum_{j=1}^NJ^{j\rightarrow i}M_t^j-M_t^i(X_T^i-X_R^i)
	\]
	Here b is Lipschitz continuous, and $\forall X, b(X)=-\lambda(X-a), (\lambda,a)\in\mathbb{R}_+$. This function will make the potential drift towards the value a. The J term models the interactions between neurons, their values depends on the number of neurons in the system and which neurons are actually involved in the interaction. \\
	Throughout the event several values of interaction have been tested. They generally are Bernoullis variables, determined at the creation of the system.\\