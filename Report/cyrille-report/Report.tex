\documentclass{report}

\usepackage{amsmath}
\usepackage{amsfonts}
\usepackage{mathtools}
% \usepackage[nottoc, notlof, notlot]{tocbibind}
\usepackage{tocbibind}
\usepackage{hyperref}
\usepackage{mathrsfs}
\usepackage{listings}
\usepackage{color}
\usepackage{caption}

\DeclarePairedDelimiter\ceil{\lceil}{\rceil}
\DeclarePairedDelimiter\floor{\lfloor}{\rfloor}

\renewcommand*\thelstnumber{\arabic{lstnumber}:}

\DeclareCaptionFormat{mylst}{\hrule#1#2#3}
\captionsetup[lstlisting]{format=mylst,labelfont=bf,singlelinecheck=off,labelsep=space}

\definecolor{mygreen}{rgb}{0,0.6,0}
\definecolor{mygray}{rgb}{0.5,0.5,0.5}
\definecolor{mymauve}{rgb}{0.58,0,0.82}

\title{CEMRACS 2017: Network of interacting neurons with random synaptic weights}
\date{October 12, 2017}
%\author{Cyrille MASCART\\ Laboratoire I3S et LJAD}
\author{}

\lstset{ %
  backgroundcolor=\color{white},   % choose the background color; you must add \usepackage{color} or \usepackage{xcolor}; should come as last argument
  basicstyle=\footnotesize,        % the size of the fonts that are used for the code
  breakatwhitespace=false,         % sets if automatic breaks should only happen at whitespace
  breaklines=true,                 % sets automatic line breaking
  captionpos=b,                    % sets the caption-position to bottom
  commentstyle=\color{mygreen},    % comment style
  deletekeywords={...},            % if you want to delete keywords from the given language
  escapeinside={\%*}{*)},          % if you want to add LaTeX within your code
  extendedchars=true,              % lets you use non-ASCII characters; for 8-bits encodings only, does not work with UTF-8
  frame=single,	                   % adds a frame around the code
  keepspaces=true,                 % keeps spaces in text, useful for keeping indentation of code (possibly needs columns=flexible)
  keywordstyle=\color{blue},       % keyword style
  language=algorithm,              % the language of the code
  morekeywords={*,...},            % if you want to add more keywords to the set
  numbers=left,                    % where to put the line-numbers; possible values are (none, left, right)
  numbersep=5pt,                   % how far the line-numbers are from the code
  numberstyle=\tiny\color{mygray}, % the style that is used for the line-numbers
  rulecolor=\color{black},         % if not set, the frame-color may be changed on line-breaks within not-black text (e.g. comments (green here))
  showspaces=false,                % show spaces everywhere adding particular underscores; it overrides 'showstringspaces'
  showstringspaces=false,          % underline spaces within strings only
  showtabs=false,                  % show tabs within strings adding particular underscores
  stepnumber=2,                    % the step between two line-numbers. If it's 1, each line will be numbered
  stringstyle=\color{mymauve},     % string literal style
  tabsize=2,	                   % sets default tabsize to 2 spaces
  title=\lstname                   % show the filename of files included with \lstinputlisting; also try caption instead of title
}

\begin{document}
\maketitle

\begin{abstract}
	Derived from the work of Hodgkin and Huxley, several examples of the fokker planck equations can be used to model the evolution in time of the potential of the membrane of a neuron
	The solutions to the Fokker-Planck equation describing the behaviour of the potential of the membrane of a neuron can sometimes degenrate and cease to exist after a finite time. The conditions for this phenomenon called blow up to exist have been studied analytically in several papers. During the 2017 session of the cemracs we have implemented an algorithm for simulating a network of neurons of potentials driven by this equation. 
	One common way to model neurons interaction in the cortex is to use a mean-field equation describing the behaviour of neurons. This kind of models have been proposed several times since the analogy between neurons and a simple electric circuitry by Hodgkin and Huxley and generalised to an equation describing the evolution of the membrane's potential of each neuron in an infinite network. In this report we are interested in describing an algorithm, based on a thinning method, that will help simulate such a network for a big number of neurons. After a description of the model, we explain in details the algorithm developed and implemented, and the choices (theoretical and practical) we made. We finish by a setting up a benchmark for experimenting with the model and showing some results obtained with the simulation.
\end{abstract}

\tableofcontents

\section{Introduction}
	One of the first models for neurons was introduced by Louis Lapicque in 1907 and called Integrate and Fire. Neurons are represented in time by the simple electrical equation
	\begin{equation}
		I(t)=C_m\frac{dV_m}{dt}
	\end{equation}
which is just the time derivative of the law of capacitance. A positive current being applied to the membrane, it's potential is going to increase until it reaches a certain threshold value $V_T$, at which point a dirac delta function happens and the voltage of the mambrane is reset to a resting potential.\\
From this model is the basis of a large variety of other neural models invented to model more precisely certain behaviours of neurons and neural networks, as memory, leaking, etc.
\section{Model and background}
	\input{Model.tex}
\section{The algorithm}
	\input{Algorithm.tex}
\section{Experiments}
	% \input{Experiments.tex}

\bibliographystyle{plain}
\nocite{*}
\bibliography{Bibliography}
\end{document}